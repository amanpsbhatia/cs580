% Template last modified by Jake Hart; please contact course staff if you have any questions regarding using this template

\documentclass{cisXXX} % You must have the cisXXX .cls file in your project or working directory (i.e. the same directory as this document) 

\HWauthor{LeBron James}{lebron@seas.upenn.edu} % Put your name and Penn email on this line
\HWno{1} % Enter the number of the homework you are working on
\HWcourse{CS 580} % Enter the course department and number here
\HWpartner{Paul Brown} % If your class allows group work, put your partners here
\HWpartner{Amelia Earhart} % Otherwise, delete or comment these lines 
\usepackage{amsmath}

\begin{document}
\maketitle
\HWproblem
This is the first problem.
This is the first problem.
This is the first problem.
This is the first problem.
This is the first problem.
This is the first problem.

This is the second paragraph of the first problem. This equation $c^2 = a^2 + b^2 - 2ab \cos(\theta_C)$ is an example of in-line math.

And this is an example of display mode math:
$$x = \frac{-b \pm \sqrt{b^2 - 4ac}}{2a}$$
This is an example of the align* environment, which can be helpful for mathematical expressions with multiple steps:
\begin{align*}
r &= 1/2 + 1/3 + 1/4 + 1/5 + 1/6 + 1/7 + 1/8 + \ldots + 1/n\\
&< 1/2 + 1/4 + 1/4 + 1/8 + 1/8 + 1/8 + 1/8 + \ldots + 1/n\\
&= \sum_{i = 1}^{\log n} \frac{2^{i - 1}}{2^i}\\
&= 1/2 \log_2 (n) = O(\log n)
\end{align*}
This is an example of matrices in \LaTeX:
\begin{align*}
	\sigma_x &=
		\begin{pmatrix}
		0 & 1 \\
		1 & 0
		\end{pmatrix} \\
	\sigma_y &=
		\begin{pmatrix}
		0 & -i \\
		i & 0
		\end{pmatrix} \\
	\sigma_z &=
		\begin{pmatrix}
		1 & 0 \\
		0 & -1
		\end{pmatrix} \\
\end{align*}
This concludes the first problem and the mathematical examples. The following questions demonstrate how to use the template for multipart problems.

\HWproblem
\HWsubproblem
This is the second problem, first question.
This is the second problem, first question.
This is the second problem, first question.
This is the second problem, first question.
This is the second problem, first question.
This is the second problem, first question.
\HWsubproblem
This is the second problem, second question.
This is the second problem, second question.
This is the second problem, second question.
This is the second problem, second question.
This is the second problem, second question.
This is the second problem, second question.
\HWsubproblem
This is the second problem, third question.
This is the second problem, third question.
This is the second problem, third question.
This is the second problem, third question.
This is the second problem, third question.
This is the second problem, third question.


\HWproblem
This is an introduction to the the third problem.
\HWsubproblem
This is the third problem, first question.
This is the third problem, first question.
This is the third problem, first question.
This is the third problem, first question.
This is the third problem, first question.
This is the third problem, first question.
\HWsubproblem
This is the third problem, second question.
This is the third problem, second question.
This is the third problem, second question.
This is the third problem, second question.
This is the third problem, second question.
This is the third problem, second question.
\HWsubproblem
This is the third problem, third question.
This is the third problem, third question.
This is the third problem, third question.
This is the third problem, third question.
This is the third problem, third question.
This is the third problem, third question.
\HWsubproblem
This is the third problem, fourth question.
This is the third problem, fourth question.
This is the third problem, fourth question.
This is the third problem, fourth question.
This is the third problem, fourth question.
This is the third problem, fourth question.

\end{document}
